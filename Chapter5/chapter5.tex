\chapter{Project Scheduling}
\begin{spacing}{1}
\setlength{\parskip}{0.3in}
\graphicspath{{./Chapter5/}}

\section{Introduction}
Project scheduling is structured listing of tasks, resources and duration  in a time sequenced manner so that the progress can be tracked and the project can be completed on time. Effective project schedule is a critical component of successful time management.
Schedule not only defines the start and end time of each activity or task, it also defines the dependencies among the tasks and the sequence of dependent tasks. A project schedule is created on the early planning phase of the project life cycle and is crucial in the project plan, where the schedule plan, schedule baseline, deliverable and requirements are identified. The project schedule is the guideline to the developer team throughout the execution phase. During the execution phase, the schedule baseline is compared against the actual progress and further modifications are added to the plan. Project scheduling is an iterative process.

\subsection{Scheduling Objectives}
A project schedule is a document that involves all the efforts needed to complete the project on time. It is also not possible to make an effective resource and cost management without a full and accurate work schedule. An effective project schedule provides a better resource and time management scheme and finds a trade off among its primary and secondary objectives.

\begin{itemize}
	\item \textbf{Primary Objectives}
	\begin{enumerate}
		\item \textbf{Best time} \newline Project schedule answers when an activity should start, how long it can take to complete and the deadline to finish the activity. 
		\item \textbf{Least cost} \newline Project schedule defines the what resources are associated with each activity and answers the lowest possible costs to complete each activity in the project.
		\item \textbf{Least risk} \newline Project schedule defines the slack and float available with each activity and dependencies among the activities to maintain.  
	\end{enumerate}
	\item \textbf{Secondary Objectives}
	\begin{enumerate}
		\item \textbf{Propose and evaluate different schedule alternatives} \newline Analyze different sequences of tasks and different schedule for their dependencies.
		\item \textbf{Make an effective use of resources} \newline Schedule the tasks in an cost effective manner so that no resources remain idle for a long period of time.
		\item \textbf{Reduce communication overhead among resources} \newline Project schedule organize the tasks in effective manner so that the resources are well distributed and available to all the components.
	\end{enumerate}
\end{itemize}

\section{Schedule Techniques}
Highly accurate estimation of duration of project tasks is the key concept to create a realistic project schedule. To make accurate estimations, project managers need to discuss with different stakeholders, review their perspectives on the tasks difficulty and possible duration, analyze previous projects and historic data. Alongside project managers follow different scheduling techniques to increase accuracy of their estimations of time and costs. 
Among different scheduling techniques, \textbf{Network Diagrams} uses to graphically represent the tasks of the project and their relationships. \textbf{CPM} and \textbf{PERT} are widely used network diagrams. The \textbf{CPM or Critical Path Method} is an equation that shows the possible longest timeline of the project. The \textbf{PERT or Program Evaluation and Review Technique} is used to visualize the flow of tasks for better estimations and also includes the dependencies.
 A \textbf{Work Breakdown Structure or WBS} is a graphical representation of decomposed tasks and deliverable of the final project. WBS becomes input to many other techniques, specially in the scheduling process. All the steps of the project are outlined in the organizational chart of a work breakdown structure.
 Another popular schedule technique is \textbf{Bar charts}. \textbf{Milestone Charts} is used to graphically represents the tasks in a temporal sequence, the bar length is directly proportional to the duration of that tasks. Another widely used technique is \textbf{Gantt Chart}. Gantt chart is also visualize the temporal sequence of tasks as milestone chart but gantt chart also shows the groups and subgroups of the tasks, dependencies among the tasks and can also shows the progress of each tasks. Gantt chart gives a more details view of the tasks sequence in temporal order.
 
 \subsection{Work Breakdown Structure}
 A work breakdown structure or WBS is visual, structured and hierarchical deconstruction of a project. 
  
\end{spacing}